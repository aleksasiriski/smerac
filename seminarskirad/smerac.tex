\documentclass[a4paper,11pt]{article}
\usepackage[a4paper, total={6in, 8in}]{geometry}
\usepackage[T1]{fontenc}
\usepackage[utf8]{inputenc}
\usepackage[english,serbian]{babel}
\usepackage{graphicx}
\usepackage{fancyhdr}

\renewcommand{\figurename}{Slika}
\graphicspath{ {./img/} }

\title{Smerač}
\author{Aleksa Siriški}
\date{Novembar 2022}

\begin{document}

\pagestyle{empty}
\begin{center}
    \begin{figure}
        \centering
        \includegraphics[height=3cm,width=3cm]{pmf}
    \end{figure}

    \textbf{
    UNIVERZITET U NOVOM SADU
    \\
    PRIRODNO-MATEMATIČKI
    \\
    FAKULTET
    \\
    DEPARTMAN ZA MATEMATIKU
    \\
    I INFORMATIKU
    }

\end{center}
\vfill
\begin{center}
	\begin{huge}
		\textbf{Smerač}
		\bigskip 
	\end{huge}
	\\
	\begin{large}
        \textbf{- seminarski rad iz predmeta Skript jezici -}
	\end{large}
\end{center}
\vfill
\begin{center}
    Aleksa Siriški, 159/22
    \\
    Novi Sad, 2022.
\end{center}
\newpage

\pagestyle{plain}
\renewcommand{\contentsname}{Sadržaj}
\addcontentsline{toc}{section}{Sadržaj}
\tableofcontents
\newpage

\pagestyle{fancy}
\fancyhf{}
\lhead{Smerač}
\rhead{Aleksa Siriški}
\cfoot{\thepage}

\section{Uvod}
\subsection{Discord}
Discord\cite{discord}, kao jedna od najpopularnijih društvenih mreža za programere i entuzijaste računarskih tehnologija, je logičan izbor za razmenu informacija i raspoređivanje časova IT i RN smerova Univerziteta u Novom Sadu. Uprkos jednostavnosti Discord-ovog grafičkog interfejsa nije lako omogućiti studentima da sami sebi određuju smer a integracija sa Google kalendarom je apsolutno neizvodljiva. Na sreću, Discord tim je osposobio trećim licima da lako kreiraju 'Bot' naloge i uz pomoć njihovog REST api će nastati Smerač - Discord Bot stvoren da omogući studentima izbor sopstvenog smera kao i integraciju Google kalendara u prostom Discord chat-u.
\subsection{Python}
Smerač je Python\cite{python} skripta napisana u svega 400 linija koda. Računajući da korisnicima nije bitno da li će im smer biti dodeljen za 1ms ili 1s Python je bio prikladan izvor. Najpre zbog mnogobrojnih ugrađenih biblioteka i modula, kao što su discordpy\cite{discordpy} i asyncio\cite{asyncio}, ali i zbog jednostavne sintakse koja je omogućila eksponencijalan razvoj ovog programa.
\subsection{Ideja}
Discord bot koji mora biti dovoljno jednostavan ali isto tako i univerzalno programiran da se može lako primeniti na više različitih okruženja (Discord servera). Takva prilagodljivost se postiže pametnim planiranjem toka rada programa i korišćenjem promenljivih okruženja (eng. \textit{environment variables}). Takođe je jako bitna licenca koja će se koristiti, da ne bi došlo do krađe autorskih prava i zloupotrebe softvera. Tačno iz tih razloga odabrana licenca za Smerač će biti GNU General Public License\cite{gpl}.
\subsection{Asinhrono izvođenje programa}
Da bi program uopšte mogao paralelno čitati poruke različitih korisnika potrebno je pozivanje novih niti za.....
\newpage

\section{Opis programa}
test
\newpage

\section{Zaključak}
test
\newpage

\pagestyle{plain}
\renewcommand\refname{Literatura}
\addcontentsline{toc}{section}{Literatura}
\bibliography{references}
\bibliographystyle{ieeetr}

\end{document}
